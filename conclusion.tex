%%%%%%%%%%%%%%%%%%%%%%
\section{Conclusion}
%%%%%%%%%%%%%%%%%%%%%%

We are in the midst of a programmable networking revolution and we believe that tackling the challenge of programmable scheduling is a key next step. FPGAs greatly facilitate this design space exploration and have been one of the key platforms used to push the boundaries of programmable networking. Schedulers have very tight timing requirements, which makes them difficult to design and build. This is the main reason why modern packet processing devices offer a small menu of fixed scheduling algorithms that can only be slightly tuned and configured. By making the scheduling decision at the time of enqueue, the PIFO abstraction leads to both an efficient implementation and a mechanism with which to program many different scheduling algorithms. As we have demonstrated, it is practical to use the PIFO to build a line-rate programmable scheduler on an FPGA which, we believe, will pave the way to further development of data-plane programming languages such as P4~\cite{p4:2014}.



